\section{Funcionamiento de Software}

El código se implementará en dos etapas, la primera consiste en una etapa de control y la segunda en una etapa de seguimiento, ambas son completamente independientes la una de la otra.

\begin{itemize}
    \item \textbf{Seguimiento}: La etapa de seguimiento implementa el sensor de línea, la idea es regular el funcionamiento de los motores de acuerdo a la corrección que ha de ser necesario para mantener el vehículo encima de la línea a seguir, y garantizar de esta forma seguimiento del trazo deseado. Toda la implementación de este código se hará programada en el IDE de Arduino puesto que al ser la etapa semi autónoma del proyecto no requiere de ningún comando desde un centro de control para la toma de decisiones. 
    Esta etapa de seguimiento implementa 3 sensores, uno para reconocer la línea a seguir, y dos sensores de obstáculo, uno de ellos encargado de detectar obstáculos frente al vehículo para su remoción, y el otro sensor para reconocer cuando se llega al final del recorrido.
    
    \item \textbf{Control}: La etapa de control se encarga de manejar el funcionamiento de los motores de manera remota mediante un módulo bluetooth, la idea es enviar datos desde un programa en C a través del puerto serial. Dependiendo de qué dato sea enviado, el vehículo llevará a cabo distintos movimientos. El programa en C cuenta con distintas funciones que permiten controlar el puerto serial, al cual se conecta el módulo bluetooth, son tres funciones en total. La primera se encarga de inicializar el puerto serial, una vez inicializado se pueden utilizar las otras dos funciones, una para escribir y la otra para leer datos, de esa forma se establece una comunicación bidireccional. Dichas funciones serán implementadas en una interfaz gráfica que simula un control remoto, la idea es controlar el vehículo con el teclado, y con la interfaz visualizar los movimientos que realiza el vehículo.
\end{itemize}




