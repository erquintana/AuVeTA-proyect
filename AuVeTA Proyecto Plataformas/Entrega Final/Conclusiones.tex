\section{Conclusiones}
\begin{itemize}
    \item Se logra comprobar con éxito el funcionamiento del vehículo tanto en el modo de control como en el modo de tracking.
    
    \item Se realizaron de manera exitosa las pruebas planteadas en la sección 3 de este documento, esto a pesar de que no se toma en consideración el cálculo de tiempo de autonomía de la fuente de poder con todos los subsistemas en funcionamiento, ya que el tiempo de operación es relativamente corto. No se descarta realizar un pruba de este punto en el futuro con todos los subsistemas trabajando al máximo ciclo de trabajo para determinar la duración de la batería ante condiciones críticas.
     
    \item Se logra la correcta implementación de la interfaz gráfica que permite la comunicación inalámbrica por conexión \textit{bluetooth} con el vehículo.
    
    \item El valor asignado a la potencia del motor para el modo tracking debe ser ajustado dependiendo de la superficie en donde trabaje el vehículo, ya que mucha potencia propiciará que el vehículo se salga de la pista y muy poca hará que aumenten las posibilidades de atascamiento.
    
    \item Para los sensores ópticos se deben considerar las condiciones de luz en el entorno de trabajo y por ello también la calibración de los mismos para garantizar un funcionamiento adecuado y óptimo de los sistemas de control.
     
    \item Se concluye  que se pueden mejorar aspectos constructivos y estéticos de la interfaz para alcanzar un nivel más depurado de experiencia para el usuario. 
\end{itemize}