\section{Reseña del programa}

El proyecto AuVeTA consiste en diseñar y poner en funcionamiento un vehículo que pueda ser utilizado como una base para prototipado de un vehículo que a futuro pueda utilizarse en aplicaciones de seguimiento de linea para algún tipo de aplicación industrial o inclusive evolucionar al mundo del reconocimiento de patrones y la visión por computador para generar una propuesta de vehículos realmente autónomos y con aplicación enfocada a la movilidad humana.

Este proyecto consiste en el desarrollo de un vehículo semi autónomo que, mediante la implementación de microcontroladores, sensores, un módulo de comunicación y programación, pueda cumplir con el objetivo de seguir una línea de color en el suelo y llegar a un punto donde culmine con la ejecución de una tarea determinada. Parte de las funciones que se desean incorporar a este prototipo, además del seguimiento de línea y remoción de obstáculos, es la comunicación del vehículo con un centro de control, desde el cual también se pueda conducir a discreción el mismo, con la finalidad de ser utilizado en otro tipo de aplicaciones mediante sus distintos modos de operación.

El proyecto consta de dos partes principales: \textit{hardware} y \textit{software}. El desarrollo e integración de estos dos subsistemas han de ir de la mano, puesto que un error en uno de ellos ha de limitar la acción correcta del otro.